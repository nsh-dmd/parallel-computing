\documentclass{article}
\usepackage[utf8]{inputenc}
\usepackage{fancyhdr} % Required for custom headers
%\usepackage{lastpage} % Required to determine the last page for the footer
\usepackage{extramarks} % Required for headers and footers
\usepackage[usenames,dvipsnames]{color} % Required for custom colors
\usepackage{graphicx} % Required to insert images
\usepackage{listings} % Required for insertion of code
\usepackage{courier} % Required for the courier font
\usepackage{lipsum} % Used for inserting dummy 'Lorem ipsum' text into the template
\usepackage{enumerate}
\usepackage{multicol}
\usepackage{caption}
\usepackage{subcaption}
\usepackage{ulem} % underline emph
\usepackage{amsmath} % for \text in mathmode
\usepackage[hypcap]{caption}

% Margins
\topmargin=-0.45in
\evensidemargin=0in
\oddsidemargin=0.5in
\textwidth=5.5in
\textheight=9.0in
\headsep=0.25in

\linespread{1.3} % Line spacing

% Set up the header and footer
\pagestyle{fancy}
\lhead{} % Top left header
\chead{\hmwkClass: \hmwkTitle} % Top center head
\rhead{\firstxmark} % Top right header
\lfoot{\lastxmark} % Bottom left footer
\cfoot{\thepage} % Bottom center footer
%\rfoot{Page\ \thepage\ of\ \protect\pageref{LastPage}} % Bottom right footer
\renewcommand\headrulewidth{0.4pt} % Size of the header rule
\renewcommand\footrulewidth{0.4pt} % Size of the footer rule

\setlength\parindent{0pt} % Removes all indentation from paragraphs

\definecolor{MyDarkGreen}{rgb}{0.0,0.4,0.0} % This is the color used for comments
\lstloadlanguages{C} % Load C syntax for listings, for a list of other languages supported see: ftp://ftp.tex.ac.uk/tex-archive/macros/latex/contrib/listings/listings.pdf
\lstset{language=C, % Use python in this example
        frame=single, % Single frame around code
        basicstyle=\small\ttfamily, % Use small true type font
        keywordstyle=[1]\color{Blue}\bf, % C functions bold and blue
        keywordstyle=[2]\color{Purple}, % C function arguments purple
        keywordstyle=[3]\color{Blue}, % Custom functions \underbar underlined and blue
        identifierstyle=, % Nothing special about identifiers                                         
        commentstyle=\usefont{T1}{pcr}{m}{sl}\color{MyDarkGreen}\small, % Comments small dark green courier font
        stringstyle=\color{Purple}, % Strings are purple
        showstringspaces=false, % Don't put marks in string spaces
        tabsize=5, % 5 spaces per tab
        %
        % Put standard Python functions not included in the default language here
        morekeywords={rand},
        %
        % Put Python function parameters here
        morekeywords=[2]{on, off, interp},
        %
        % Put user defined functions here
        morekeywords=[3]{glutCreateWindow,p},
        %
        morecomment=[l][\color{Blue}]{...}, % Line continuation (...) like blue comment
        numbers=none, % can use none % Line numbers on left
        firstnumber=1, % Line numbers start with line 1
        numberstyle=\tiny\color{Blue}, % Line numbers are blue and small
        stepnumber=1 % Line numbers go in steps of 5
}
% \usepackage{graphicx}
\newcommand{\indep}{\rotatebox[origin=c]{90}{$\models$}}

% Creates a new command to include a perl script, the first parameter is the filename of the script (without .pl), the second parameter is the caption
\newcommand{\code}[1]{
\begin{itemize}
\item[]\lstinputlisting[label=#1]{#1.c}
%\item[]\lstinputlisting[caption=#2,label=#1]{#1.c}
\end{itemize}
}

%----------------------------------------------------------------------------------------
%   DOCUMENT STRUCTURE COMMANDS
%   Skip this unless you know what you're doing
%----------------------------------------------------------------------------------------

\setcounter{secnumdepth}{0} % Removes default section numbers

\newcommand{\homeworkProblemName}{}
\newenvironment{homeworkProblem}[1]{ % Makes a new environment called homeworkProblem which takes 1 argument (custom name) but the default is "Problem #"
    \renewcommand{\homeworkProblemName}{#1} % Assign \homeworkProblemName the name of the problem
    \section{\homeworkProblemName} % Make a section in the document with the custom problem count
}

\newcommand{\problemAnswer}[1]{ % Defines the problem answer command with the content as the only argument
    \noindent\framebox[\columnwidth][c]{\begin{minipage}{0.98\columnwidth}#1\end{minipage}} % Makes the box around the problem answer and puts the content inside
}

\newcommand{\homeworkSectionName}{}
\newenvironment{homeworkSection}[1]{ % New environment for sections within homework problems, takes 1 argument - the name of the section
    \renewcommand{\homeworkSectionName}{#1} % Assign \homeworkSectionName to the name of the section from the environment argument
    \subsection{\homeworkSectionName} % Make a subsection with the custom name of the subsection
}

%----------------------------------------------------------------------------------------
%   NAME AND CLASS SECTION
%----------------------------------------------------------------------------------------

\newcommand{\hmwkTitle}{Problem set 3\\ Theory} % Assignment title
\newcommand{\hmwkDueDate}{\date{Oktober 05, 2016}} % Due date
\newcommand{\hmwkClass}{TDT4200} % Course/class
\newcommand{\hmwkAuthorName}{Neshat\ Naderi}  % Your name


%----------------------------------------------------------------------------------------
%   TITLE PAGE
%----------------------------------------------------------------------------------------

\title{
\vspace{2in}
\textmd{\textbf{\hmwkClass:\ \hmwkTitle}}\\
\normalsize\vspace{0.1in}\normalsize{\hmwkDueDate}
\vspace{0.1in}\large{\text{Parallel Computing}}
\vspace{3in}
}

\author{\textbf{\hmwkAuthorName}}
\date{} % Insert date here if you want it to appear below your name

%----------------------------------------------------------------------------------------
\begin{document}
\maketitle

\setcounter{tocdepth}{1} % Uncomment this line if you don't want subsections listed in the ToC

% \newpage
% \tableofcontents
%\newpage

%----------------------------------------------------------------------------------------
%   PROBLEM 1
%----------------------------------------------------------------------------------------

% To have just one problem per page, simp   ly put a \clearpage after each problem
\clearpage

\section{Part 1, Theory}
\subsection{Problem 1}
\begin{flushleft}
a) Cache is basically memory locations which are accessed much faster than main memory. Cache memory is also called CPU cache and is physically located either onchip or in a place where CPU can access that very fast.

\end{flushleft}
\begin{flushleft}
b) First of all locality is a principle implying that accessing a location of cache is followed by accessing a nearby location. So \textbf{spatial} locality happens when a memory block is accessed then a program can access a nearby location. So in spatial locality memory locations are accessed close in both time and space. But if some memory location is accessed several times and in a relativly near future it is called \textbf{temporal} locality.
\end{flushleft}

\begin{flushleft}
c) Cache coherence is read and write behaviors regarding consistency of the cache memory. Cache coherence problems often happens when the memory is shared or resources are stored in different parts of cache memory.
\end{flushleft}

\begin{flushleft}
d) Multiple cores, shared memory systems can have a behavior called \textbf{false sharing}. It happens when multiple processors modify shared data simultanously, updating a cache line at a same time or a location on cache line is updated very frequently. \\ 
The cache is organized in cache lines and each core has its own cache usually. When a cache miss happesn when a processor reads from an address not existing in its cache, the whole cache line is read from main memory. In this case this cache line is marked invalid and all other copies in other processors' cache are also invalid. False sharing ocuures when cores updates a different locations in the memory but in the same cache line and it is called false because they act like they share memory while they actually do not.
\end{flushleft}
\newpage

\subsection{Problem 2}
a)
\begin{lstlisting}

#include <semaphore.h>

void Send_msg(void∗ rank) {
    long my_rank = (long) rank;
    long dest = (my rank + 1) % thread count;
    char* my_msg = malloc(MSG-MAX * sizeof(char));

    sprintf(my_msg, "Hello to %1d from %1d, dest, my_rank");
    messages[dest] = my_msg;

    sem_post(&semaphores[dest]);
        /* Unlock the semaphore of dest */

    /* Wait for our semaphoreto be unlocked */    
    sem_wait(&semaphores[my_rank]);
    sprintf("thread %1d > %s\n", my_rank, messages[my_rank]);

}

\end{lstlisting}
\newpage
b) In this listing if 
\begin{lstlisting}

#include <semaphore.h>

int counter = 0;
sem_t count_sem = 1;
sem_t barrier_sem = 0;

void Thread_work(void∗ rank) {
    . . .

    /* Barrier */
    sem_wait(&count_sem);

    if( counter == thread_count - 1) {
        counter = 0;
        sem_post(&count_sem);
        for(int i = 0; i < thread_count - 1; i++) {
            sem_post(&barrier_sem);
        }
        else {
            counter++;
            sem_post(&count_sem);
            sem_wait(&barrier_sem);
        }
        . . .
    }
}
\end{lstlisting}


\subsection{Problem 3}
\begin{flushleft}
a) OpenMP is a highlevel language extension and can parallelize programes with simple directives. While Pthread is an API with alot of functions which is compatible with all C compilers. It is more lowlevel and requires explicit specifying of each threads behavior. But in OpenMP some of those behaviors are determined by the compiler and run-time system.
\end{flushleft}
\begin{flushleft}
b) 
\begin{lstlisting}
# pragma omp parallel for 
for(int i = 0; i < n; i++) {
    calculate(i);
}
\end{lstlisting}
\end{flushleft}

\end{document}